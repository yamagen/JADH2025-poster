\newif\ifENG
\ENGtrue % 英語版を使用する場合はtrue、和文版を使用する場合はfalseに設定

\documentclass{article}
\usepackage[paperwidth=84.1cm, paperheight=118.9cm, margin=0cm]{geometry}
\usepackage{tikz}
\usepackage{hyperref}
\usepackage{graphicx}
\usepackage{fontspec}

% 色の定義
\definecolor{mydeepblue}{HTML}{332E83}

% 背景用
\definecolor{deepBackground}{HTML}{332E83} % 深い藍(背景など)
\definecolor{softGray}{HTML}{CCCCCC}       % 淡い灰色(背景補助)
\definecolor{lightIvory}{HTML}{F8F5E1}     % 生成的な淡ベージュ

% タイトル・見出し用
\definecolor{katanoPurple}{HTML}{4B0082}   % 目立つ紫
\definecolor{katanoBlue}{HTML}{4466CC}     % 明るい青
\definecolor{katanoIndigo}{HTML}{3A4F92}   % 落ち着いた藍
\definecolor{katanoBrown}{HTML}{5C4033}    % 和風の焦げ茶

\definecolor{katanoMediumOrchid}{HTML}{BA55D3} % 明るい紫、ピンクがかる
\definecolor{katanoMediumPurple}{HTML}{9370DB} % 少し青みが強い
\definecolor{katanoOrchid}{HTML}{DA70D6}       % さらに明るいピンク寄り
\definecolor{katanoViolet}{HTML}{EE82EE}       % 淡い紫、明度が高い
\definecolor{katanoSlateBlue}{HTML}{6A5ACD} % くっきりめ紫
\definecolor{katanoPurpleStrong}{HTML}{6A5ACD} % SlateBlue系


\definecolor{katanoSlateBlue}{HTML}{6A5ACD} % 青紫・視認性良好
\definecolor{katanoMediumSlateBlue}{HTML}{7B68EE} % やや明るい青紫
\definecolor{katanoBlueViolet}{HTML}{8A2BE2} % かなり青い鮮やか紫

% 本文用
\definecolor{katanoTextGray}{HTML}{333333} % 濃い灰(本文)

\setmainfont{Noto Sans}
\newfontfamily\jpfont{Noto Sans CJK JP}

%\setmainfont{Noto Sans CJK JP} % 日本語フォントを指定

\pagestyle{empty}

\begin{document}

\begin{tikzpicture}[remember picture, overlay]

% 背景色
%\fill[blue] (current page.south west) rectangle (current page.north east);

% 背景画像(使う場合コメントアウトを外す)
%\node[anchor=south west, inner sep=0] at (current page.south west) {
%  \includegraphics[width=\paperwidth,height=\paperheight]{background.jpg}
%};

% 背景画像をページ全面に敷く
\node[
  anchor=south west,
  inner sep=0,
  opacity=0.2, % 透明度を設定
] at (current page.south west) {
  \includegraphics[width=\paperwidth,height=\paperheight]{
%  ise-katano.jpg
    images/yatsuhashi.jpg
}
};

% 画像の著作権
\node[
  anchor=north west,
  align=left,
  color=katanoMediumPurple,
  font=\fontsize{20pt}{20pt}\bfseries
] at ([xshift=1cm,yshift=2cm]current page.south west) {
\ifENG
% Background: Katano; Admiring the Scattered Cherry Blossoms
Background Image courtesy of the Museum of Fine Arts, Boston; Public Domain
%  \jpfont{画像提供:\textbf{いせかたの}(伊勢物語のカタノ)}
\else
画像出典:ボストン美術館(パブリックドメイン)
\fi
};

% Title
\node[
  anchor=north,
  align=center,
  text=blue!80!black,
  font=\fontsize{60pt}{70pt}\bfseries
] at ([yshift=-2cm]current page.north) {
  Translating Ise Monogatari through the Lens of Process Grammar Model
};

% Subtitle
\node[
  anchor=north,
  align=center,
  text=blue!80!black,
  font=\fontsize{60pt}{70pt}\bfseries
] at ([yshift=-5cm]current page.north) {
  A Bilingual and Structured Approach to Classical Japanese Narratives
};

% Authors
\node[
  anchor=north,
  align=center,
  font=\fontsize{50pt}{40pt}\bfseries
] at ([yshift=-8cm]current page.north) {
Hilofumi Yamamoto{\raisebox{1.5ex}{\fontsize{20pt}{20pt}\selectfont 1}}
\quad
Bor Hodošček{\raisebox{1.5ex}{\fontsize{20pt}{20pt}\selectfont 2}}
\quad
Xudong Chen{\raisebox{1.5ex}{\fontsize{20pt}{20pt}\selectfont 1}}
};

% 所属
\node[
  anchor=north,
  align=center,
  font=\fontsize{40pt}{30pt}\bfseries
] at ([yshift=-11cm]current page.north) {
{\raisebox{1.5ex}{\fontsize{20pt}{20pt}\selectfont 1}}
Institute of Science Tokyo
\quad
{\raisebox{1.5ex}{\fontsize{20pt}{20pt}\selectfont 2}}
The University of Osaka
};


\node[
  draw=katanoBlue,
  line width=3pt,
  inner sep=10mm, 
  anchor=north west,
  text width=35cm,
  align=left,
  text=mydeepblue,
  font=\fontsize{30pt}{20pt}\bfseries
] at ([xshift=4cm,yshift=-15cm]current page.north west) {
  \textbf{\fontsize{40pt}{30pt}\selectfont Introduction}\\[1em]
  \textcolor{katanoBrown}{
    \begin{itemize}
      \item Ise Monogatari: A classic of Japanese literature
      \item Process Grammar Model: A framework for understanding narrative structures
      \item Bilingual approach: Bridging classical Japanese and modern English
    \end{itemize} 
  }
};


% 左上テキストボックス
\node[
  draw=katanoBlue,
  line width=3pt,
  inner sep=10mm, 
  anchor=north west,
  text width=35cm,
  align=left,
  text=mydeepblue,
  font=\fontsize{30pt}{20pt}\bfseries
] at ([xshift=4cm,yshift=-27cm]current page.north west) {
  \textbf{\fontsize{40pt}{30pt}\selectfont Problem}\\[1em]
  \textcolor{katanoBrown}{
    \begin{itemize}
      \item Translation used to write unwritten elements in the original text.
      \item Translation has not been provided the intention of the author's unwritten meaning.
      \item Translation does not allow readers to speculate the author's intention.
      \item The original work would be less interesting by reading the such translation.
   \end{itemize} 
  }
};

% 右上テキストボックス
\node[
  draw=katanoBlue,
  line width=3pt,
  inner sep=10mm, 
  anchor=north east,
  text width=35cm,
  align=left,
  text=mydeepblue,
%  font=\fontsize{30pt}{20pt}\bfseries
  font=\jpfont\fontsize{30pt}{20pt}\bfseries
] at ([xshift=-3cm,yshift=-15cm]current page.north east) {
  \textbf{\fontsize{40pt}{30pt}\selectfont Methods}\\[1em]
  \textcolor{katanoBrown}{
    \begin{itemize}
      \item 4-Step Translation Process
      \item Use of Process Grammar Model
%      \item gloss の本質は「原語を知らなくても、意味と文法機能が理解できる」記述にある。
      \item The essence of gloss is to provide a description that allows understanding of meaning and grammatical function without knowing the original language.
%      \item その点で、あなたの設計方針は正確で、本質を突いています。
      \item In that respect, this design philosophy is accurate and gets to the essence.
    \end{itemize} 
  }
};


% ariwara no narihira ason
\node[
  draw=katanoBlue,
  line width=0pt,
  inner sep=10mm, 
  anchor=north east,
  text width=12cm,
  align=center,
  text=mydeepblue,
  font=\jpfont\fontsize{20pt}{10pt}\bfseries
] at ([xshift=-3cm,yshift=-7cm]current page.north east) {
  \includegraphics[trim=110 90 90 310,clip,width=12cm]{images/ariwaranonarihira.jpg}
  \\[.5em]
  \textbf{\fontsize{27pt}{10pt}\selectfont Ariwara no Narihira}
  \includegraphics[trim=0 0 0 0,clip,width=1.2cm]{images/PD-icon.png}
};

% 中央下テキスト
\node[
  draw=katanoBlue,
  line width=3pt,
  inner sep=10mm, 
  anchor=north west,
  text width=75cm,
  align=left,
  text=mydeepblue,
  font=\fontsize{30pt}{20pt}\bfseries
] at ([xshift=4cm, yshift=-50cm]current page.north west) {
  \textbf{\fontsize{40pt}{30pt}\selectfont Results and Conclusion}\\[1em]
  \textcolor{katanoBrown}{
  \begin{itemize}
    \item Enhanced understanding of Ise Monogatari
    \item Effective bilingual translation
    \item Insights into narrative structures
    \item Check the flow of the translation process
  \end{itemize}
  }
};

% 中央下右テキスト
\node[
  draw=katanoBlue,
  line width=3pt,
  inner sep=10mm, 
  anchor=north west,
  text width=75cm,
  align=left,
  text=mydeepblue,
  font=\fontsize{30pt}{20pt}\bfseries
] at ([xshift=4cm, yshift=-111cm]current page.north west) {
  \textbf{\fontsize{40pt}{30pt}\selectfont References}\\[1em]
  - Yamamoto, H. (2025). Process Grammar Model (v1.0.11). Zenodo. 
  \href{https://doi.org/10.5281/zenodo.15613134}{
    \texttt{https://doi.org/10.5281/zenodo.15613134}
    \includegraphics[height=0.9cm]{images/zenodo.15613134.png}
  }

};


% ロゴ(右下)
\node[
  anchor=south east
] at ([xshift=-1cm,yshift=1cm]current page.south east) {
  \includegraphics[height=4cm]{images/sciencetokyo.png}
  \includegraphics[trim=30 30 30 30, clip, height=4cm]{images/theUnivOsaka.jpg}
};

\end{tikzpicture}

\end{document}



