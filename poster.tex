\newif\ifENG
\ENGtrue % 英語版を使用する場合はtrue、和文版を使用する場合はfalseに設定

\documentclass{article}
\usepackage[paperwidth=84.1cm, paperheight=118.9cm, margin=0cm]{geometry}
\usepackage{tikz}
\usepackage{graphicx}
\usepackage{fontspec}

% 色の定義
\definecolor{mydeepblue}{HTML}{332E83}

% 背景用
\definecolor{deepBackground}{HTML}{332E83} % 深い藍(背景など)
\definecolor{softGray}{HTML}{CCCCCC}       % 淡い灰色(背景補助)
\definecolor{lightIvory}{HTML}{F8F5E1}     % 生成的な淡ベージュ

% タイトル・見出し用
\definecolor{katanoPurple}{HTML}{4B0082}   % 目立つ紫
\definecolor{katanoBlue}{HTML}{4466CC}     % 明るい青
\definecolor{katanoIndigo}{HTML}{3A4F92}   % 落ち着いた藍
\definecolor{katanoBrown}{HTML}{5C4033}    % 和風の焦げ茶

% 本文用
\definecolor{katanoTextGray}{HTML}{333333} % 濃い灰(本文)




\setmainfont{Noto Sans}
\newfontfamily\jpfont{Noto Sans CJK JP}

%\setmainfont{Noto Sans CJK JP} % 日本語フォントを指定

\pagestyle{empty}

\begin{document}

\begin{tikzpicture}[remember picture, overlay]

% 背景色
%\fill[blue] (current page.south west) rectangle (current page.north east);

% 背景画像(使う場合コメントアウトを外す)
%\node[anchor=south west, inner sep=0] at (current page.south west) {
%  \includegraphics[width=\paperwidth,height=\paperheight]{background.jpg}
%};

% 背景画像をページ全面に敷く
\node[
  anchor=south west,
  inner sep=0,
  opacity=0.2, % 透明度を設定
] at (current page.south west) {
  \includegraphics[width=\paperwidth,height=\paperheight]{
%  ise-katano.jpg
    yatsuhashi.jpg
}
};

% 画像の著作権
\node[
  anchor=north west,
  align=left,
  color=blue,
  font=\fontsize{20pt}{20pt}\bfseries
] at ([xshift=1cm,yshift=2cm]current page.south west) {
\ifENG
% Background: Katano; Admiring the Scattered Cherry Blossoms
Background Image courtesy of the Museum of Fine Arts, Boston; Public Domain
%  \jpfont{画像提供:\textbf{いせかたの}(伊勢物語のカタノ)}
\else
画像出典:ボストン美術館(パブリックドメイン)
\fi
};

% ここに他のノードを追加


% タイトル
\node[
  anchor=north,
  align=center,
%  color=mydeepblue, 
%  text=katanoPurple,
  text=katanoBlue,
  font=\fontsize{60pt}{70pt}\bfseries
] at ([yshift=-2cm]current page.north) {
  Translating Ise Monogatari through the Lens of Process Grammar Model
};

% サブタイトル
\node[
  anchor=north,
  align=center,
  font=\fontsize{60pt}{70pt}\bfseries
] at ([yshift=-5cm]current page.north) {
  A Bilingual and Structured Approach to Classical Japanese Narratives
};

% 著者
\node[
  anchor=north,
  align=center,
  font=\fontsize{50pt}{40pt}\bfseries
] at ([yshift=-8cm]current page.north) {
Hilofumi Yamamoto{\raisebox{1.5ex}{\fontsize{20pt}{20pt}\selectfont 1}}
\quad
Bor Hodošček{\raisebox{1.5ex}{\fontsize{20pt}{20pt}\selectfont 2}}
\quad
Xudong Chen{\raisebox{1.5ex}{\fontsize{20pt}{20pt}\selectfont 1}}
};

% 所属
\node[
  anchor=north,
  align=center,
  font=\fontsize{40pt}{30pt}\bfseries
] at ([yshift=-11cm]current page.north) {
{\raisebox{1.5ex}{\fontsize{20pt}{20pt}\selectfont 1}}
Institute of Science Tokyo
\quad
{\raisebox{1.5ex}{\fontsize{20pt}{20pt}\selectfont 2}}
The University of Osaka
};


% 左上テキストボックス
\node[
  anchor=north west,
  text width=35cm,
  align=left,
  color=blue,
  font=\Large
] at ([xshift=2cm,yshift=-12cm]current page.north west) {
  \textbf{Background}\\[1em]
  This section describes the background of the research...
};

% 右上テキストボックス
\node[
  anchor=north east,
  text width=35cm,
  align=left,
  color=blue,
  font=\Large
] at ([xshift=-2cm,yshift=-12cm]current page.north east) {
  \textbf{Methodology}\\[1em]
  Here you explain your methods in detail...
};

% 中央下テキスト
\node[
  anchor=north,
  text width=70cm,
  align=left,
  color=blue,
  font=\Large
] at ([yshift=-65cm]current page.north) {
  \textbf{Results and Conclusion}\\[1em]
  Summarize the results and main findings here...
};

% ロゴ(右下)
\node[
  anchor=south east
] at ([xshift=-1cm,yshift=1cm]current page.south east) {
  \includegraphics[height=7cm]{sciencetokyo.png}
  \includegraphics[trim=30 30 30 30, clip, height=7cm]{theUnivOsaka.jpg}
};

\end{tikzpicture}

\end{document}



\documentclass[a0]{beamer}
\usepackage[size=a0,scale=1.2,orientation=portrait]{beamerposter}
\usepackage{tikz}
\usetikzlibrary{arrows.meta,decorations.pathmorphing}
\usepackage{luatexja}
\usepackage{luatexja-fontspec}
%\setmainjfont{IPAexGothic}
\setmainjfont{Noto Sans CJK JP}




\title{Translating Ise Monogatari through the Lens of Process Grammar Model: A Bilingual and Structured Approach to Classical Japanese Narratives}
\author{Hilofumi Yamamoto$^1$ \and Bor Hodošček$^2$ \and Xudong Chen$^1$}
\institute{Institute of Science Tokyo \and The University of Osaka \and Institute of Science Tokyo}
\date{2025}


\setbeamercolor{title}{fg=white,bg=blue}
\setbeamerfont{title}{series=\bfseries}

\begin{document}
\begin{frame}[t]


\institute{Institute of Science Tokyo \and The University of Osaka \and Institute of Science Tokyo}
\date{2025}


\setbeamercolor{title}{fg=white,bg=blue}
\setbeamerfont{title}{series=\bfseries}

\begin{document}
\begin{frame}[t]


\begin{tikzpicture}[remember picture,overlay]
\coordinate (base) at (current page.south west);
\node at (current page.center) {\Huge \textcolor{red}{$\rightarrow$}};
\node at (10cm,-10cm) {ここに置く};
\node at ([xshift=15cm,yshift=20cm] base) {左下基準でここ};
\node at ([xshift=11cm,yshift=11cm] base) {\Huge \textcolor{red}{(0,0)}};
\end{tikzpicture}

\begin{beamercolorbox}[wd=\textwidth,sep=1em,center]{title}

{\fontsize{64pt}{60pt}\selectfont\textbf{Translating Ise Monogatari through the Lens of Process Grammar Model:\\A Bilingual and Structured Approach to Classical Japanese Narratives}}
\end{beamercolorbox}



\begin{beamercolorbox}[wd=\textwidth,sep=0.5em,center]{author}
{\Large
Hilofumi Yamamoto\textsuperscript{1}
\quad
Bor Hodošček\textsuperscript{2}
\quad
Xudong Chen\textsuperscript{1}
}
\end{beamercolorbox}

\begin{beamercolorbox}[wd=\textwidth,sep=0.5em,center]{institute}
{\large
\textsuperscript{1} Institute of Science Tokyo
\quad
\textsuperscript{2} The University of Osaka
}
\end{beamercolorbox}


\begin{columns}[t]

\begin{column}{0.9\textwidth}
\begin{beamercolorbox}[wd=\textwidth]{block title}
縦レイアウトテスト
\end{beamercolorbox}
\begin{beamercolorbox}[wd=\textwidth]{block body}
このポスターはA0縦方向に設定しています。
PDFビューアで表示すると横向きに見える場合がありますが、印刷時に縦で出力されます。
\end{beamercolorbox}
\end{column}

\end{columns}

\end{frame}
\end{document}
